% TODO: DA SOSTITUIRE CON IL FRONTESPIZIO UFFICIALE
\Universita{University of Trento}
\CorsoDiLaurea{Master's Degree in Data Science}
\AnnoAccademico{Academic Year 2023--2024 [FIRST PAGE TO BE SUBSTITUTED WITH THE OFFICIAL FIRST PAGE AFTER EXPORTING THE DOCUMENT]}
\Titolo{RAG-based Personalization and LLMs Evaluation for AI Chatbots}
\Relatore{Prof. Jacopo Staiano}
\RelatoreLabel{Supervisor}
\CandidatoLabel{Graduate Student}

\Candidato{Enrico Zanetti} 
\Matricola{238680}
\DataEsame{September 17, 2024}
\Logo{layout/unitn-logo.jpg}
\LogoWidth{7cm} %optional, default: 3cm
\LogoPosition{top}
\opacitaSfondo{0.05}

\begin{titlepage}
    \newgeometry{left=3cm, right=3cm, bottom=2cm, top =3cm} 
    \pagestyle{empty}
    \makefrontpage
    \restoregeometry
\end{titlepage}

\frontmatter
% dedica
~ \newpage
\null\vspace{\stretch{1}}
\begin{flushright}
    \textit{Inserire dedica/citazione qui.}
\end{flushright}
\vspace{\stretch{2}}\null

%ringraziamenti
\chapter*{Acknowledgments}

\section*{Motivation}
% DA RISCRIVERE ALLA FINE
My interest in this research stems from a deep passion for natural human languages and the transformative potential of artificial intelligence (AI) applications. Additionally, my motivation is deeply rooted in my fascination with the complexities of human societies and global phenomena, a passion that also stems from my background in International Studies. Over the past few years, during my master's studies in Data Science, I developed a profound interest in the technical aspects of AI, particularly the intricacies of machine learning and deep learning. I am particularly captivated by how AI can radically improve and simplify people's lives in essential tasks such as information retrieval. The rapid advancements in AI, especially in the development of AI chatbots and large language models (LLMs), have heightened my interest due to their profound implications for how people interact with technology and access information. The potential of AI to enhance human life, making tasks more efficient and interactions more meaningful, continues to inspire my research and drive my curiosity.

However, alongside these technical fascinations lie concerns for the ethical and societal risks associated with AI. The challenges that AI systems face, such as bias, misinformation, and ethical dilemmas, highlight the need for a responsible approach to AI development. Moreover, the technical limitations of AI, including issues related to model size, the intricacies of training processes, and the evaluation of AI performance, further underscore the complexity of ensuring these systems are both effective and ethically sound. My curiosity extends beyond the technical aspects of AI to the geopolitical, ethical, and regulatory dimensions that influence how these technologies are governed and integrated into society.

In particular, I am drawn to the complexities of evaluating and improving AI systems, especially in terms of personalization and reliability. The innovative techniques like Retrieval-Augmented Generation (RAG) present a promising avenue to address some of these challenges, offering a way to enhance the accuracy and relevance of AI-generated content. My research is motivated by a desire to contribute to this evolving field, exploring solutions that can mitigate the risks of AI while maximizing its benefits for users worldwide. 

Through this work, I aim to bridge the gap between technological innovation and ethical responsibility, ensuring that AI not only advances in capability but also aligns with the values and needs of society.

\begin{abstract}

This thesis explores AI chatbot development landscape with the integration of Large Language Models (LLMs) and Retrieval-Augmented Generation (RAG), emphasizing the dual objectives of enhancing user interaction and evaluating model performance. It begins by tracing the evolution of AI chatbots, from early rule-based systems to advanced LLM-driven interfaces, highlighting their applications across various domains such as education, healthcare, and finance. The study delves into the foundational aspects of LLMs, including their architecture, training methodologies, and limitations, with a particular focus on the Transformer architecture and the RAG paradigm. RAG, which combines retrieval mechanisms with generative models, is analyzed for its potential to improve the relevance and accuracy of chatbot responses, thereby enabling greater personalization.

The thesis delves into the complexities and methodologies involved in evaluating LLMs and RAG systems, proposing new metrics and benchmarks specifically designed for these advanced models. It features an in-depth case study on WISE, an AI chatbot developed by HPA, illustrating the real-world impact of integrating the RAG framework into AI-driven applications. Additionally, the thesis introduces the LLM Evaluator, detailing its methodology, implementation, testing and visualization. The analysis of both applications includes also a discussion of their respective limitations. Furthermore, the thesis addresses ethical considerations and regulatory frameworks, emphasizing the critical importance of responsible AI development in the evolving global landscape.

Overall, this research analyzes the advancements of personalized AI chatbots by integrating LLMs and RAG technology, providing novel insights into model evaluation, and proposing future directions for both technological innovation and ethical governance in AI.

\end{abstract}

\tableofcontents

\printglossaries
\addcontentsline{toc}{chapter}{Glossary}

\printnomenclature
\addcontentsline{toc}{chapter}{Nomenclature list}

\mainmatter

% --- NOTE: REMEMBER TO SORT IT IN ALPHABETICAL ORDER ---

\section*{Glossary}
\begin{description}
    \item[AI] Artificial Intelligence, the simulation of human intelligence in machines.
    \item[API] Application Programming Interface, a set of tools and protocols for building software and applications.
    \item[CNN] Convolutional Neural Network, a type of deep learning algorithm primarily used for processing structured grid data like images.
    \item[FFNN] Feed-Forward Neural Network, a type of neural network where connections between the nodes do not form cycles.
    \item[FLOP] Floating Point Operations, a measure of computational performance, often used to quantify the operations in machine learning models.
    \item[KG] Knowledge Graph, a network of entities and their relationships, used to represent and store complex structured and unstructured data.
    \item[LLM] Large Language Model, a type of AI model designed to understand and generate human-like text.
    \item[ML] Machine Learning, a subset of AI that focuses on building systems that learn from data.
    \item[NLP] Natural Language Processing, a field of AI that focuses on the interaction between computers and humans through natural language.
    \item[PEFT] Parameter-Efficient Fine-Tuning, a method that fine-tunes models using fewer parameters, making the process more efficient and less resource-intensive.
    \item[R\&D] Research \& Development, activities that companies undertake to innovate and introduce new products and services.
    \item[RAG] Retrieval-Augmented Generation, a framework that combines retrieval mechanisms with generative models to improve the accuracy and relevance of AI-generated responses.
    \item[RLHF] Reinforcement Learning from Human Feedback, a training methodology that improves models based on feedback from human evaluators.
    \item[SSO] Single sign-on, an authentication method enabling a user to log in with a single ID across multiple related, independent software systems.
\end{description}
