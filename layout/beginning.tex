% TODO: DA SOSTITUIRE CON IL FRONTESPIZIO UFFICIALE
\Universita{University of Trento}
\CorsoDiLaurea{Master's Degree in Data Science}
\AnnoAccademico{Academic Year 2023--2024 [FIRST PAGE TO BE SUBSTITUTED WITH THE OFFICIAL FIRST PAGE AFTER EXPORTING THE DOCUMENT]}
\Titolo{RAG-based Personalization and LLMs Evaluation for AI Chatbots}
\Relatore{Prof. Jacopo Staiano}
\RelatoreLabel{Supervisor}
\CandidatoLabel{Graduate Student}

\Candidato{Enrico Zanetti} 
\Matricola{238680}
\DataEsame{September 17, 2024}
\Logo{layout/unitn-logo.jpg}
\LogoWidth{7cm} %optional, default: 3cm
\LogoPosition{top}
\opacitaSfondo{0.05}

\begin{titlepage}
    \newgeometry{left=3cm, right=3cm, bottom=2cm, top =3cm} 
    \pagestyle{empty}
    \makefrontpage
    \restoregeometry
\end{titlepage}

\frontmatter
% dedica
~ \newpage
\null\vspace{\stretch{1}}
\begin{flushright}
    \textit{Inserire dedica/citazione qui.}
\end{flushright}
\vspace{\stretch{2}}\null

%ringraziamenti
\chapter*{Acknowledgments}

\section*{Motivation}

My interest in this research stems from a deep passion for natural human languages and the transformative potential of artificial intelligence (AI) applications. In particular, I am fascinated by how AI can radically improve and simplify people's lives in essential tasks such as information retrieval. As AI technologies have advanced, the capabilities of chatbots and language models have grown dramatically, offering unprecedented opportunities to improve user interactions and access to information.

However, despite these advances, AI chatbots still face significant challenges. Problems such as hallucination, in which models generate plausible but incorrect information, and bias embedded in training data that can lead to biased or incorrect results, are major concerns. In addition, the static nature of many models means that they are not updated with the latest information, limiting their effectiveness over time.

To address these challenges, my thesis explores innovative solutions such as Retrieval Augmented Generation (RAG) and advanced evaluation techniques. RAG combines the strengths of large language models with retrieval mechanisms to ensure that responses are based on up-to-date and accurate information. This approach reduces hallucination and increases the reliability of the results generated.

In addition, evaluation and monitoring of performance of AI chatbots are critical to understanding and mitigating bias. By systematically analyzing model performance and incorporating continuous feedback loops, we can develop more robust and fair artificial intelligence systems. This thesis delves into these methodologies, providing a comprehensive analysis of the state of the art of current techniques and their practical applications.

\begin{abstract}
This thesis analyzes the evolving landscape of chatbot personalization and evaluation, focusing on the core technology: Large Language Models (LLMs). The paper is divided into several key sections, beginning with an introduction and a comprehensive literature review examining related work and chatbot personalization. 

The development and techniques section delves into the architecture and training processes of LLMs, including model size, pre-training, next-token and multi-token prediction, and refinement methods such as Low-Rank Adaptation (LoRA) and prompt-based learning. This section also covers augmented LLMs, focusing on augmented retrieval generation frameworks, and evaluation and tracking of LLMs, highlighting the importance of tracking ML experiments and tools such as MLtraq.

State-of-the-art models are discussed, addressing their challenges, limitations and advances over traditional transformers, particularly the LLM Mamba architecture.

A significant portion of the thesis is devoted to a case study analysis of WISE and the LLM evaluator. This includes an in-depth analysis of WISE's High Performance Analytics (HPA), its role in document management transformation, its capabilities, benefits, industry applications and future enhancements. The HPA LLM evaluator is also examined, covering methodology, implementation, testing, visualization, creation of a Python package and its limitations.

The final sections of the thesis address ethical and regulatory considerations related to the deployment and use of LLMs.

Overall, this thesis aims to provide a detailed understanding of LLMs, their development, evaluation, and practical applications, while also considering the ethical and regulatory landscape.

\end{abstract}

\tableofcontents

\printglossaries
\addcontentsline{toc}{chapter}{Glossary}

\printnomenclature
\addcontentsline{toc}{chapter}{Nomenclature list}

\mainmatter
