\section{Related Works}
Review of related work focusing on LLM-based chatbots in healthcare, education, customer-support, finance, business.

\section{Introduction}

The exponential increase in data generation has profoundly transformed the digital information landscape, with the total volume of data created, captured, copied, and consumed globally reaching approximately 120 zettabytes in 2023. This number will rise to 147 zettabytes by 2024 with expectations to surpass 181 zettabytes by 2025 \cite{taylor2023volume}. This rapid expansion of the data ecosystem has catalyzed significant advancements in artificial intelligence (AI), particularly in the development of large language models (LLMs) \cite{zhao2023survey}. LLMs, known for their exceptional ability to comprehend, generate, and manipulate human language \cite{brown2020language}, have become a cornerstone in the evolution of AI chatbots \cite{dam2024complete}.

% insert data graph. First page of dam2024complete

In the era of AI-driven chatbots, LLMs have emerged as pivotal tools, powering conversational capabilities and enabling human-like interactions \cite{koubaa2023exploring}. The surge in data, coupled with advancements in computational techniques, has significantly enhanced the functionality of LLM-based chatbots, making them indispensable across various sectors. These chatbots' ability to understand and respond with unprecedented contextual relevance and accuracy, while managing vast streams of information, has rendered them crucial in domains such as education, research, healthcare, and many others \cite{dam2024complete}.

Given the vast potential and the expanding utilization of LLM-based chatbots, there is a pressing need for thorough research and evaluation to optimize their performance. This necessity becomes even more apparent as the domain of LLM-based chatbots rapidly grows, leading to an overwhelming volume of research literature that demands comprehensive analysis. Consequently, this chapter provides a timely and thorough review of LLM-based chatbots, addressing their development, applications across various industries, key challenges, and strategies for enhancing their effectiveness and reliability. % to fix what will cover this chapter

\section{Evolution of AI Chatbots and LLMs}

Before the advent of LLMs, conversational AI encountered numerous challenges, such as limited contextual understanding and domain specificity, which often resulted in inaccurate and robotic responses. Early chatbots lacked sophisticated language comprehension, leading to disjointed and non-human-like user interactions. Moreover, scalability across different industries was problematic, particularly in handling vast information streams with real-time responsiveness. However, the introduction of LLMs marked a revolutionary shift in chatbot technology, inaugurating a new era of AI-driven interactions \cite{dam2024complete}.

% --------------- ARRIVED HERE ---------------
% 1. Take trend_comparison and create a plot.
% 2. Go on with dam2024complete and write the section: Pre-LLM Era of Chatbots
% 3. Always in dam2024 write the APPLICATIONS: the core of this chapter

In March 2023, OpenAI launched GPT-4, following the widespread success of ChatGPT 3.5, released in November 2022 \cite{bahrini2023chatgpt, zhang2023one}. The exponential rise in popularity of ChatGPT and AI, as illustrated in Figure \ref{fig:chatgpt_popularity}, underscores its dominance over other emerging technologies such as 5G, Bitcoin, and Blockchain \cite{tech_trends_2023}. This trend highlights a new chapter in AI-driven communication. In response, Google introduced BARD, its first LLM-based chatbot, in early 2023, further contributing to the growing landscape of LLM-powered conversational agents \cite{google_bard_2023}.

\section{Existing Surveys, Reviews, and Case Studies}

Numerous articles have reviewed the wide-ranging applications of LLM-based chatbots, emphasizing their significant impact and the complex challenges they pose across various sectors. This section discusses some of these articles and demonstrates how our survey extends and differentiates itself from them.

\subsection{Educational Impact of AI and Chatbots}

Several studies have explored the influence of AI and chatbots in the academic domain, particularly concerning their ethical implications and transformative potential in education. For instance, \cite{ai_education_2023} examines the ethical challenges posed by AI and chatbots in academic research and assessment, proposing solutions to mitigate potential misuse in educational settings. Another study \cite{chatgpt_online_learning_2023} focuses on how ChatGPT enhances online learning, with findings indicating that students appreciate the interactive and engaging environment these agents provide.

\subsection{Healthcare Applications of ChatGPT}

In the healthcare sector, the role of LLM-based chatbots has also been extensively reviewed. \cite{chatgpt_healthcare_2023} provides a systematic analysis of ChatGPT's impact on healthcare, particularly in medical education, research, and clinical practice. The study highlights the potential of ChatGPT to revolutionize scientific writing and personalized learning while addressing concerns regarding ethical considerations and accuracy.

\subsection{Cross-Disciplinary Reviews}

Beyond education and healthcare, other reviews have evaluated ChatGPT's impact across multiple disciplines. For example, \cite{chatgpt_multidisciplinary_2023} examines its applications in fields ranging from marketing to environmental science, while also discussing the ethical challenges and practical deployment issues. Similarly, \cite{chatgpt_impact_2023} offers a detailed analysis of ChatGPT's performance across subjects like economics, programming, and law, emphasizing the need for updated assessment methods and educational policies.

\section{Our Contribution}

In contrast to existing works that often focus on specific chatbots like ChatGPT, our survey extends the scope to cover a broader range of models, including BARD, Bing Chat, and Claude. We explore their applications across multiple domains, addressing various challenges, each discussed in detail under several sub-categories. This comprehensive approach not only provides a deeper understanding of LLM-based chatbots but also offers valuable insights for future research.

\section{Chatbot personalization}
Evolution of chatbot personalization. Previous methods: Example-Based Dialogue Management (EBDM) and ML-based approaches for personality prediction.
