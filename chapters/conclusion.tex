In this thesis, the journey from the foundational concepts of data-driven models to the sophisticated architecture of large language models (LLMs) has been explored, with a specific focus on the evolution and development of AI chatbots. The progression from early, rule-based systems to the advanced AI-driven chatbots of today highlights the transformative potential of these technologies in various sectors, including education, healthcare, and finance.

The technical foundations underpinning these advancements, particularly the Transformer architecture and its associated mechanisms such as self-attention and multi-head attention, have been thoroughly examined. The exploration into the impact of model size, training processes, and the latest fine-tuning techniques has provided insights into how these elements contribute to the performance and adaptability of LLMs. Moreover, the limitations of LLMs in practical applications were critically analyzed, emphasizing the challenges of bias, misinformation, and the need for ongoing innovation in the field.

One of the significant contributions of this work is the exploration of Retrieval-Augmented Generation (RAG) as a method to enhance personalization in AI chatbots. The detailed analysis of RAG, including indexing optimization, retrieval source, and query optimization, showcases how this technique can improve the relevance and accuracy of responses generated by AI systems. The future prospects of RAG technology were also discussed, highlighting its potential to address some of the current limitations of LLMs.

The evaluation of LLMs and RAG systems presented in this thesis underscored the importance of developing robust evaluation metrics and benchmarks tailored to these models. The challenges in evaluating these systems were addressed, providing a framework for assessing their effectiveness and fairness in real-world applications.

A case study on WISE, an AI chatbot developed by High Performance Analytics (HPA), demonstrated the practical implementation of RAG-based personalization in a real-world scenario. The analysis of the LLM Evaluator further emphasized the importance of comprehensive testing, visualization, and the creation of specialized tools to support the evaluation process.

Finally, this thesis has underscored the ethical considerations and regulatory frameworks essential to the responsible development and deployment of AI technologies. The discussion on global AI regulations, including those in the European Union, United States, and China, highlighted the varying approaches and the critical need for policies that balance innovation with societal safeguards.

In conclusion, this thesis has provided a comprehensive exploration of the technical, practical, and ethical dimensions of AI chatbot development and LLM evaluation. The insights gained through this research contribute to the ongoing discourse on AI technologies, offering pathways for future advancements that are both innovative and aligned with ethical standards.
