The rapid advancement of artificial intelligence (AI) has revolutionized various sectors, leading to the proliferation of AI-powered applications that significantly impact individual lives, businesses and broader societal structures. Among these applications, AI chatbots, driven by Large Language Models (LLMs), have emerged as prominent tools in natural language processing (NLP), offering sophisticated interaction capabilities that mimic human conversation. The impact of LLMs is further underscored by the remarkable uptake of applications like ChatGPT. Upon its release in late November 2022, ChatGPT quickly gained widespread popularity, reaching one million users within just five days \cite{wu2023brief}. By February 2023, it had achieved a record-breaking growth trajectory, amassing 100 million users in only two months. In 2024, ChatGPT generates on average 1.7 billion monthly site views, reflecting the profound appeal and utility of this technology across various domains \cite{aiprm2023chatgpt}. Besides ChatGPT, the development of these chatbots has seen a remarkable evolution, transitioning from rudimentary rule-based systems to complex AI-driven models capable of performing a wide range of tasks across different domains.

This thesis explores the intersection of personalized AI chatbots and the evaluation of LLMs, with a particular focus on the integration of Retrieval-Augmented Generation (RAG) techniques. The objective is to investigate how RAG can enhance personalization in chatbots, making interactions more relevant and user-centric, while also addressing the challenges associated with evaluating the performance and ethical implications of these advanced models.

The thesis is structured into several chapters, each delving into key aspects of AI chatbot development, LLMs, and RAG techniques. Chapter 2 provides a comprehensive overview of the evolution of AI chatbots, tracing their journey from the early days of data-driven models to the sophisticated LLM-based systems we see today. It then transitions to the modern applications of AI chatbots across various industries, including education, research, healthcare, software engineering, and finance. Each of these sectors has seen significant advancements in how chatbots are utilized to improve efficiency, enhance user experiences, and drive innovation. The discussion of this chapter ended with the exploration of how early attempts at personalization were limited by the technology of the time, and how modern approaches, particularly those leveraging LLMs, have dramatically improved the ability of chatbots to deliver tailored and contextually relevant interactions.

Chapter 3 explores the technical foundations and innovations driving AI development, with a focus on LLMs and their role in the evolution of chatbots. The chapter begins by tracing the history of chatbots and LLMs, from early rule-based systems in the 1960s to the sophisticated AI chatbots of today, highlighting key milestones such as the introduction of Transformer architecture in 2017.

The discussion delves into the critical components of Transformer architecture, including positional encoding, self-attention mechanisms, and multi-head attention, which have enabled significant advancements in model performance. It also examines the impact of model size on AI capabilities and tracks recent trends in the expansion of parameter sizes.

Further sections cover essential processes like pre-training, fine-tuning, and prompt-based learning, including advanced methods like Parameter Efficient Fine-Tuning (PEFT) and Low-Rank Adaptation (LoRA). The chapter also introduces Retrieval-Augmented Generation (RAG) as a technique to enhance the relevance and accuracy of AI responses, discussing its components such as indexing optimization, query processing, and generation methods.

The chapter concludes by addressing the limitations of LLMs in practical applications and considers future prospects for RAG technology and developments beyond the current Transformer-based models.

Chapter 4 delves into the critical aspects of evaluating LLMs and their performance in various applications. The chapter begins with a general overview of LLM evaluation, discussing the challenges and importance of assessing these models’ accuracy, adaptability, and effectiveness. It then explores specific metrics and methodologies used in evaluating Natural Language Generation (NLG), including model-based evaluation metrics and NLP tasks, LLM-derived metrics, and the use of prompting techniques to enhance evaluation accuracy.

The chapter also addresses the unique challenges associated with evaluating Retrieval-Augmented Generation (RAG) systems, which combine the generative capabilities of LLMs with real-time data retrieval. RAG-specific evaluation metrics are discussed in detail, along with the challenges that arise when assessing the effectiveness of these hybrid models. Additionally, the chapter outlines various benchmarks for LLM and RAG systems, focusing on general tasks, specific downstream tasks, and multi-modal tasks. By examining both the success and failure cases of LLMs, the chapter provides a nuanced understanding of where these models excel and where they fall short. The chapter concludes with a discussion on the future directions for LLM evaluation, emphasizing the need for continuous improvement and innovation in this rapidly evolving field.

Chapter 5 presents a case study analysis, focusing on two key applications: WISE and the LLM Evaluator. The chapter begins by introducing WISE, a sophisticated AI chatbot developed by HPA, and examines its functionality, features, and technological stack. WISE is analyzed in the context of its real-world applications, particularly in transforming document management processes. The chapter discusses the advantages of WISE, including its ability to streamline operations, improve user interactions, and deliver personalized experiences. Additionally, it explores the industry applications of WISE, highlighting how it has been adapted for use in various sectors.

The second part of the chapter introduces the LLM Evaluator, a tool designed to assess the performance of LLMs in different contexts. The methodology behind the LLM Evaluator is explained in detail, including its implementation, testing, and visualization processes. The chapter also discusses the limitations of both WISE and the LLM Evaluator, providing a balanced view of their strengths and areas for improvement. The chapter concludes by outlining potential future works and improvements for these applications, emphasizing the importance of continuous development and innovation.

Finally, Chapter 6 addresses the critical ethical and regulatory issues surrounding AI development, particularly in the context of LLMs and AI chatbots. The chapter begins by exploring the ethical risks associated with AI, including bias, misinformation, and the potential for unintended consequences. It discusses the limitations of LLMs, focusing on the non-existential risks they pose and the importance of addressing these issues to ensure the responsible deployment of AI technologies.

The chapter then provides an overview of global AI regulations, examining how different regions approach the governance of AI. It covers the European Union’s leadership in AI regulation, particularly through the AI Act and related initiatives, as well as the regulatory frameworks in the United States and China. The discussion highlights the challenges and opportunities presented by these regulations, emphasizing the need for a balanced approach that fosters innovation while safeguarding against potential harms. The chapter concludes by considering the future of AI regulation, calling for a collaborative effort to develop comprehensive and effective policies that can guide the ethical and responsible use of AI technologies.

This thesis aims to contribute to the ongoing discourse on AI chatbot development by offering a nuanced understanding of RAG-based personalization and the critical aspects of evaluating LLMs. Through this exploration, the research seeks to inform future advancements in AI technology, ensuring that AI chatbots continue to evolve in a manner that is both innovative and ethically sound.
